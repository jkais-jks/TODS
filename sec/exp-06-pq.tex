\subsection{Group-based Algorithm of \ours}
\noindent{\bf Hyperparamaters in LSH.} We test the performance of varying the number of hyperplanes in LSH. As shown in Figure~\ref{fig:pq-exp} (a), with the number increasing, more clusters are generated, and the tuples within each cluster are closer, so the  accuracy increases at the beginning. Afterwards, the accuracy remains stable because items in each cluster are similar enough for gradient approximation. Therefore, empirically, using about 64 hyperplanes is appropriate because more clusters will reduce the efficiency.

\noindent{\bf Hyperparamaters in PQ.}In Section~\ref{subsec:pq}, we use product quantization (PQ) to compute the upper bound $\hat{s}_{ju}$ of $\overline{s}_{ju}$. Recap that \ours$^+$ needs the user-specified cluster centers size $R$ as input, which is important for pre-computing the distance between tuples. Thus, we discuss how to select an appropriate $R$. We adopt a simple yet effective solution that select different $R$ and get different coresets. Then we train over these coresets and evaluate via a validation set to get different results. To be specific, we select $R$ from 32 to 256 for each dataset. As shown in Figure~\ref{fig:pq-exp} (b), the performance on dataset \hr, \adult, \bike and \imdbl when varying the cluster centers size $R$. We can see that We can see that as $R$ increases, the accuracy of the dataset also gradually increases, because when $R$ increases, the upper bound $\hat{s}_{ju}$ is closer to $\overline{s}_{ju}$, which can help us to select a good coreset.


We also test the performance of varying $\frac{m}{M}$ for PQ. In Figure~\ref{fig:pq-exp}(c)-(d), at the beginning,  $\frac{m}{M}=1$ means that in each subspace, the length of all sub-vectors is 1. With $\frac{m}{M}$ increasing, the accuracy increases first because each sub-vector is longer, which keeps more information when adding up these $\hat{s}^l_{it}$, leading to a more precise bound. But if each sub-vector is too long, which means that each vector is quantized to a very short code, the accuracy decreases because in this situation, the PQ method is not informative enough to give accurate Euclidean distance estimation. Empirically, $M\approx3$ is always an appropriate choice.
